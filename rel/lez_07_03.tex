\section{Lezione del 07/03/2018}

\subsection{Approfondimento sull'induzione}

\subsubsection{Induzione ordinaria} 

Proprietà $P(n)$, e.g., $ = $ ``Se $n$ è pari, $n+1$ è dispari'' oppure 
``tutti i grafi con $n$ nodi \dots ''.\par
Per dimostrare che $P(n)$ vale per ogni $n$
\begin{itemize}
	\item $P(0)$: \textbf{caso base};
	\item assumo vera $P(n) \rightarrow $ dimostro $P(n+1)$, allora $P(n)$
	è vera per ogni $n$.
\end{itemize}

\subsubsection{Induzione completa}
\begin{itemize}
	\item $\big[ P(0) \big]$ (non necessaria, è un'istanza del passo successivo);
	\item dimostro $P(m) \ \forall m<n \rightarrow $ vale $P(n) \ \forall n$.  
\end{itemize}

\subsection{Complessità di Merge Sort}

$n = \#$elementi da ordinare\footnote{Il simbolo \# verrà usato per indicare la cardinalità di un insieme.}

\paragraph{Merge(A,p,q,r)}
\begin{itemize}
	\item[] \textbf{inizializzazione}: $a'n+b'$;
	\item[] \textbf{ciclo}: $a'n+b'$;
\end{itemize}

Sommandoli, ottengo una complessità all'incirca di:
\begin{displaymath}
	T^{merge}(n) = an+b
\end{displaymath}

Nel dettaglio: 

\[ T^{MS}(n) =
\begin{cases}
	c_0       & \quad \text{se } n \leq 1 \\
	T^{MS}(n_1)+T^{MS}(n_2)+T^{merge}(n)  & \quad \text{altrimenti}
\end{cases}
\]

\begin{center}
	$\Downarrow$ \\
\end{center}

\[ T^{MS}(n) =
\begin{cases}
c_0       & \quad \text{se } n \leq 1 \\
T^{MS}(n_1)+T^{MS}(n_2)+an+b  & \quad \text{altrimenti}
\end{cases}
\]

con 

\begin{itemize}
	\item[] $T^{MS}(n_1) = \floor{\frac{n}{2}}$
	\item[] $T^{MS}(n_2) = \ceil{\frac{n}{2}}$
\end{itemize}

%\subsubsection{Complessità }

\begin{align*}
	T^{MS}(n) \\
	an+b
\end{align*}
\begin{align*}
	T^{MS}(n_1) && T^{MS}(n_2) \\
	an_1+b && an_2+b
\end{align*}
\begin{align*}
T^{MS}(n_{11}) && T^{MS}(n_{12}) && T^{MS}(n_{21}) && T^{MS}(n_{22}) \\
an_{11}+b && an_{12}+b && an_{21}+b && an_{22}+b
\end{align*}
\begin{center}
	\dots
\end{center}
\begin{align*}
	c_0 && c_0 && \dots && \dots && \dots && c_0 && c_0 \\
\end{align*}

Otteniamo $c_0$ ripetuto $n$ volte all'ultimo livello dell'albero. Vediamo nel dettaglio la complessità
nelle varie iterazioni.
\begin{itemize}
	\item[] $\boldsymbol{i=0}$ \hspace{0.5cm} $an+b$
	\item[] $\boldsymbol{i=1}$ \hspace{0.5cm} $a(n_1+n_2)+2b \approx an+2b$
	\item[] $\boldsymbol{i=2}$ \hspace{0.5cm} $a(n_{11}+n_{12}+n_{21}+n_{22})+4b \approx an+4b$
	\item[] \dots
	\item[] $\boldsymbol{i=h}$ \hspace{0.5cm} $c_0n$
\end{itemize}

Poniamo $n=2^h$. Abbiamo

\begin{align*}
	T^{MS}(n) 	& = \displaystyle\sum_{i=0}^{h-1}(an+2^ib)+c_0n \\
				& = anh+b\displaystyle\sum_{i=0}^{h-1}2^i && \text{($h= \log_2n$)}\\
				& = an \log_2n + b2^h - b + c_0n && \text{($2^h=n$)} \\
				& = an \log_2n + (b+c_0)n - b
\end{align*}
\begin{displaymath}
	T^{MS}(n) = an \log_2n + b''n + c'' \approx n \log_2n
\end{displaymath}

\subsection{Confronto tra IS e MS}

\begin{gather*}
	T^{IS}(n) = a'n^2 + b'n + c' \\
	T^{MS}(n) = an \log_2n + b''n + c''
\end{gather*}

Posso calcolare il limite del rapporto:
\begin{displaymath}
	\lim_{n \to +\infty} \frac{T^{MS}(n)}{T^{IS}(n)} = \lim_{n \to +\infty} \frac{an \log_2n + b''n + c''}{a'n^2 + b'n + c'} = 0
\end{displaymath}

Per definizione
\begin{gather*}
	\forall \varepsilon > 0 \ \exists n_0 : \forall n \geq n_0 \quad \frac{T^{MS}(n)}{T^{IS}(n)} < \varepsilon \\
	\Downarrow
\end{gather*}
\begin{align*}
	T^{MS}(n) < \varepsilon T^{IS}(n) = \frac{T^{IS}}{m} && \text{(Ponendo, ad esempio, $\varepsilon = \frac{1}{m}$)}
\end{align*}

Detto a parole, c'è un certo $n$ oltre il quale, ad esempio, \texttt{Merge Sort} su un \emph{Commodore 64} 
esegue più velocemente di un \texttt{Insertion Sort} su una macchina moderna, come mostrato nella seguente 
tabella.
\begin{center}
	\begin{tabular}{l|c|c}
		$n$ & $T^{IS}(n)=n^2$ & $T^{IS}(n)=n \log n$ \\
		\hline
		10 & $0.1ns$ & $0.033ns$ \\
		\hline
		1000 & $1ms$ & $10\mu s$ \\
		\hline
		$10^6$ & 17 minuti & $20ms$ \\
		\hline
		$10^9$ & 70 anni & $30s$ \\
	\end{tabular}
\end{center}